\documentclass{enpc-article}

\title{\vspace{-2em}Analyse de Fourier et Applications : Séance 2}
\author{KI Club info des Ponts}
\date{\today}

\def\ens{L^1([-\pi, \pi], \mathbb{C})}

\begin{document}
\maketitle

\section*{Exercice 1.3}

Pour $u \in \ens$, $c_n(u)$ est bien définie et à valeurs dans $\mathbb{C}$.
On a aussi $c_n(u) \le \frac{1}{\sqrt{2 \pi}} \norm{u}_{L^1}$ donc $c_n$ est une forme linéaire continue de $\ens$ dans $\mathbb{C}$.

Soit $n \in \mathbb{N}$. On a :
\begin{align*}
    c_n(\overline{u}) &= \frac{1}{\sqrt{2 \pi}} \int_{-\pi}^{\pi} \overline{u}(x) \exp(- i n x) dx \\
    &= \frac{1}{\sqrt{2 \pi}} \int_{-\pi}^{\pi} \overline{u}(x) \exp(i (-n) x) dx \\
    &= \frac{1}{\sqrt{2 \pi}} \int_{-\pi}^{\pi} \overline{u}(x) \overline{\exp(-i (-n) x)} dx \\
    &= \frac{1}{\sqrt{2 \pi}} \overline{\int_{-\pi}^{\pi} u(x) \exp(-i (-n) x) dx} \\
    &= \overline{c_{-n}(u)}
\end{align*}

Pour le deuxième résultat, on effectue le changement de variable $y = -x$ :
\begin{align*}
    c_n(\tilde{u}) &= \frac{1}{\sqrt{2 \pi}} \int_{-\pi}^{\pi} u(-x) \exp(- i n x) dx \\
    &= \frac{1}{\sqrt{2 \pi}} \int_{-\pi}^{\pi} u(y) \exp(i n y) dy \\
    &= \frac{1}{\sqrt{2 \pi}} \int_{-\pi}^{\pi} u(y) \exp(- i (-n) y) dy \\
    &= c_{-n}(u) \\
\end{align*}

\end{document}
