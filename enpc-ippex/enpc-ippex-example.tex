\documentclass{enpc-ippex}

\firstname{John}
\lastname{Doe}
\teacher{M. Jean-Yves Poitrat}
\master{Mme. Jane Doe}
\title{Découverte du métier d'employé commercial dans la grande distribution}
\date{30 septembre 2016}
\period{lundi 6 juin 2016 -- samedi 2 juillet 2016}
\place{Carrefour Champs-sur-Marne}
\location{Avenue des Pyramides \\ 77420 Champs-sur-Marne}

\begin{document}
\maketitle

\tableofcontents

\part*{Introduction}
\addcontentsline{toc}{part}{Introduction}
Ce stage peut être considéré comme intéressant et représentatif quand on considère que le commerce à prédominance alimentaire est un important secteur économique, avec 12 291 points de vente, plus de 2 500 points de retrait « drive », ainsi que 602 000 salariés en 2013 \cite{forco}.

\part{La routine, source d'aliénation au travail}

\part{Le management, ou son absence, au service de l'humanisation du travail}

\part{Le recours aux stagiaires, un travail à la rencontre d’intérêts divergents}

\begin{chapquote}{\textbf{Décret n° 2015-1359 du 26 octobre 2015 relatif à l'encadrement \\ du recours aux stagiaires par les organismes d'accueil}}
Le nombre de stagiaires dont la convention de stage est en cours pendant une même semaine civile ne peut excéder 15 \% de l'effectif pour les organismes d'accueil dont l'effectif est supérieur ou égal à vingt
\end{chapquote}

\part*{Conclusion}
\addcontentsline{toc}{part}{Conclusion}

%\bibliographystyle{ieeetr}
% \bibliography{biblio}

\pagebreak

\appendix

\pagebreak

\part*{Questionnaire}

\section*{À remplir par l’étudiant :}

\paragraph{Intitulé du poste :}

\paragraph{Principales tâches effectuées :}

\paragraph{Horaires de travail :}

\paragraph{Nom de l’enseignant :}

\section*{À remplir par l’enseignant}


\paragraph{Analyse qualitative du travail fourni par l’élève :}

\paragraph{Investissement dans le stage :}
\vspace{1.5cm}

\paragraph{Qualité de l’observation et de la réflexion :}
\vspace{1.5cm}

\paragraph{Qualité de la restitution :}
\vspace{1.5cm}

\begin{itemize}
    \item à l’oral :
    \vspace{1.5cm}
    \item à l’écrit :
    \vspace{1.5cm}
\end{itemize}

% \end{Form}

\end{document}
